\documentclass[11pt]{article}
)
\usepackage[T2A]{fontenc}     % внутренняя T2A кодировка TeX
\usepackage[russian]{babel}   % включение переносов
\usepackage[utf8]{inputenc}
\usepackage[margin=0.4in]{geometry}
\pagestyle{empty} % нумерация выкл.
\addtolength{\textheight}{1.75in}
\usepackage[colorlinks]{hyperref}
\usepackage{longtable}
\usepackage{color}
\usepackage{setspace}
\usepackage{multirow}
\usepackage{graphicx}

% Изменить заголовок списка литературы
\addto\captionsrussian{\def\refname{Publications and resources}}

\makeatletter
\bibliographystyle{unsrt}
\renewcommand{\@biblabel}[1]{#1.} 
\makeatother

\renewenvironment{itemize}{
	\begin{list}{\labelitemi}{
			\setlength{\topsep}{0pt}
			\setlength{\partopsep}{0pt}
			\setlength{\parskip}{0pt}
			\setlength{\parsep}{0pt}
			\setlength{\itemsep}{0pt}
		}
	}{\end{list}}

\begin{document}	
	\noindent {\sffamily{\Huge{\textbf{Nikita Shubin}}}}
	\vspace{0.5em}
	
	\noindent +7 920 633 94 11 | \href{mailto:shubin.kit@ya.ru}{shubin.kit@ya.ru} | Ryazan, Russia (relocation is acceptable)
	\vspace{0.5em}
	\hrule
	
	%\vspace{1.5em}
	
	%\noindent {\textbf{Цель:}} Участие в разработке нейроподобных интеллектуальных систем. Возможность получить глубокое представление о принципах работы отдельных нейронов и нервной системы вцелом. Перевод биологических принципов самоорганизации на язык математики и программирования. Создание сильного искусственного интеллекта.
	
	\vspace{1em}
	
	\noindent {\textbf{Profile:}} Team Lead/Data Techlead/Senior Data Scientist (DL, CV). \href{https://disk.yandex.ru/i/ffl8gXNlYxGT8g}{PhD in engineering}. General work experience as a research programmer -- 18 years.\\

	%\vspace{1em}
	\noindent {\textbf{Employment history:}}
	
	\begin{tabular} {l | p{0.815\textwidth}}
		\textbf{2023 -- Now} & ML Team Lead in \href{https://agrodigit.ru/}{<<ЦПТ <<Агроцифра>> LLC}:
		\begin{itemize}
			\item Built a video data processing pipeline for dataset expansion.
			\item Assembled an annotation team from scratch.
			\item Developed software that drastically simplifies the data annotation process in \href{https://www.cvat.ai/}{CVAT}, utilizing pre-annotation with the Segment Anything Model \href{https://segment-anything.com/}{v1} and  \href{https://sam2.metademolab.com/demo}{v2}.
			\item Developed and trained models to solve assigned problems.
		\end{itemize}\\
		\textbf{2022 -- 2023} & Head of Machine Learning Team in <<ЭДС>> LLC:
		\begin{itemize}
			\item Object detection and segmentation on photos and videos (NDA).
			\item Combined several data sources for training.
			\item I was the product owner of a synthetic dataset generator based on \href{https://www.unrealengine.com/en-US/unreal-engine-5}{Unreal Engine 5}, as well as filling the real dataset in CVAT.
		\end{itemize}\\
		\textbf{2021 -- 2022} & Video Analytics Team Lead (Senior) in \href{https://www.stream-labs.com/}{Stream Labs}:
		\begin{itemize}
			\item a neural network model for ECG classification was developed, which is now working in the system called \href{https://va1235.wixsite.com/tis-tat/edinyj-kardiolog-respubliki-tatarst}{«Unified cardiologist of the Tatarstan Republic»};
			\item a prototype of pathology detection in lungs computed tomography has been developed.
		\end{itemize}\\
		\textbf{2020 -- 2021} & Research programmer in \href{https://www.spiritdsp.com/company/}{Spirit DSP}:
		\begin{itemize}
			\item the development of a new video codec based on neural networks;
			\item developing a real-time neural network replacing the background for video communication.
		\end{itemize}\\
		\textbf{2007 -- 2020} & Software engineer in Department of Automation and Information Technologies in control of \href{http://www.rsreu.ru/en/}{Ryazan State Radio Engineering University}:
		\begin{itemize}
			\item My research on image analysis have received \href{https://www.rfbr.ru/project_search/350568/}{RFBR}, \href{https://grants.extech.ru/grants/res/winners.php?OZ=9&TZ=K&year=2016}{Grant from the President of Russia} and \href{http://rscf.ru/sites/default/files/docfiles/Winners_0029.pdf}{RSF} grant support;
			\item \href{https://ryazan.bezformata.com/listnews/aspiranti-rgrtu-pobediteli-konkursa/81548415/}{«Young scientist of the year 2020»};
			\item \hyperlink{AutorIDs}{40 publications in conference proceedings and scientific journals};
			\item 2 patents and 1 computer program registration certificate.
		\end{itemize}
		\vspace*{-\baselineskip} % Удаление отсупа после
	\end{tabular}
	\vspace{0.9em}\\
	\begin{longtable} {r | p{0.77\textwidth}}
		\textbf{Area of interest}\vspace{1em} & СV, ML, DL, multi-agent systems, embadded systems.\\
		
		\textbf{Research directions}\vspace{1em} & Object detection and parameters estimation algorithms: classical methods of image processing \cite{wrt}, neural network technologies application for programming the behavior of agents in a multi-agent system \cite{mas}. Efficient video data compression.\\
		
		\textbf{Basic skills} & \underline{Programming}: \it{Python, Matlab, Bash (Linux), Docker, CVAT, \LaTeX, Git.}\vspace{0.5em}\\ 
		& \underline{ML stack}: \it{Keras, TensorFlow, PyTorch, KerasTuner, ONNX, OpenCV, Scikit-learn, Scikit-image, Albumentations, Jupyter, Numpy, Numba, Pandas.}\vspace{0.5em}\\
		\vspace{1em} & \underline{Scientific research results publication}: \it{Participation in conferences, articles preparation.}\\
		
		\textbf{Additional skills} & \underline{3D graphics:} \it{3DS Max, Blender, mathematical foundations of computer graphics.}\\
		& \underline{Image/video/audio processing/labeling:} \it{Photoshop, Premier Pro, VirtualDub, ffmpeg, SAM, GroundingDINO.}\\
		\vspace{1em} & \underline{English level:} \it{Intermediate.}\\
	\end{longtable}		
	
	\newpage
	Appendix
	\hrule
	\vspace{1em}
	
	\noindent {\textbf{Education:}}
	
	\begin{tabular} {c | c | p{0.62\textwidth}}
		\textbf{2003 -- 2008} & \href{http://www.rsreu.ru/en/}{RSREU} & \href{http://www.rsreu.ru/en/about-university/faculties-and-departments/faculty-of-automation-and-information-technologies-in-sontrol}{Faculty of Automation and Information Technologies in Сontrol.} \vspace{0.5em}\\
		\textbf{2008 -- 2011} & Postgraduate studies & Department of Automation and Information Technologies in Сontrol.\\
	\end{tabular}
	
	\begin{longtable} {r | p{0.77\textwidth}}
		\textbf{Self-education}
		& 
		\href{https://lab.karpov.courses/certificate/752f8aad-7f6e-40ad-98c2-12efe12bc0b7/en/}{Course «System Design»}\\
		&
		\href{https://disk.yandex.ru/i/H5-s3Lw6j-1wkg}{Professional retraining} course \href{https://disk.yandex.ru/i/UO1qC0H4FCa8LQ}{\textquotedblleft Data science, neural networks, machine learning and artificial intelligence\textquotedblright} \cite{inpainting} and \href{https://disk.yandex.ru/i/t3yNB56C3nNo5A}{\textquotedblleft Advanced course\textquotedblright}\\
		&  \href{https://www.coursera.org/account/accomplishments/certificate/DMUZAUCWBQG4}{Course «Stats for data analysis»}\\
		& \href{https://www.coursera.org/account/accomplishments/certificate/H8BWKAK4Y96D}{Course «Unsupervised learning»}\\
		& \href{https://www.coursera.org/account/accomplishments/certificate/9GQC5C4UP9YB}{Course «Supervised learning»}\\
		& \href{https://www.coursera.org/account/accomplishments/certificate/2AZHZW96LJ2N}{Course «Mathematics and python for data analysis»}\\
		\vspace{1em} & 
		\href{https://www.coursera.org/account/accomplishments/certificate/6K5Z2UFA5887}{Course «Machine Learning Introduction»}\\
		
		\hypertarget{AutorIDs}{\textbf{Author IDs}} & WoS ResearcherID: \href{https://publons.com/researcher/2345963/nikita-y-shubin/}{\underline{N-5016-2015}}\\
		& Scopus AuthorID: \href{https://www.scopus.com/authid/detail.uri?authorId=56094972000}{\underline{56094972000}}\\
		& Orcid: \href{https://orcid.org/0000-0003-4563-5643}{\underline{0000-0003-4563-5643}}\\
		& Google Scholar Citations ID: \href{https://scholar.google.com/citations?user=auKREHMAAAAJ}{\underline{auKREHMAAAAJ}}
	\end{longtable}
	
	%\addcontentsline{toc}{section}{Список ресурсов}
	\bibliography{refs_eng}

\end{document}